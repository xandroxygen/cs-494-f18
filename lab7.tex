\documentclass[]{article}
\usepackage{lmodern}
\usepackage{amssymb,amsmath}
\usepackage{ifxetex,ifluatex}
\usepackage{fixltx2e} % provides \textsubscript
\ifnum 0\ifxetex 1\fi\ifluatex 1\fi=0 % if pdftex
  \usepackage[T1]{fontenc}
  \usepackage[utf8]{inputenc}
\else % if luatex or xelatex
  \ifxetex
    \usepackage{mathspec}
  \else
    \usepackage{fontspec}
  \fi
  \defaultfontfeatures{Ligatures=TeX,Scale=MatchLowercase}
\fi
% use upquote if available, for straight quotes in verbatim environments
\IfFileExists{upquote.sty}{\usepackage{upquote}}{}
% use microtype if available
\IfFileExists{microtype.sty}{%
\usepackage{microtype}
\UseMicrotypeSet[protrusion]{basicmath} % disable protrusion for tt fonts
}{}
\usepackage[margin=1in]{geometry}
\usepackage{hyperref}
\hypersetup{unicode=true,
            pdftitle={Lab 7: Clustering in R},
            pdfborder={0 0 0},
            breaklinks=true}
\urlstyle{same}  % don't use monospace font for urls
\usepackage{color}
\usepackage{fancyvrb}
\newcommand{\VerbBar}{|}
\newcommand{\VERB}{\Verb[commandchars=\\\{\}]}
\DefineVerbatimEnvironment{Highlighting}{Verbatim}{commandchars=\\\{\}}
% Add ',fontsize=\small' for more characters per line
\usepackage{framed}
\definecolor{shadecolor}{RGB}{248,248,248}
\newenvironment{Shaded}{\begin{snugshade}}{\end{snugshade}}
\newcommand{\KeywordTok}[1]{\textcolor[rgb]{0.13,0.29,0.53}{\textbf{#1}}}
\newcommand{\DataTypeTok}[1]{\textcolor[rgb]{0.13,0.29,0.53}{#1}}
\newcommand{\DecValTok}[1]{\textcolor[rgb]{0.00,0.00,0.81}{#1}}
\newcommand{\BaseNTok}[1]{\textcolor[rgb]{0.00,0.00,0.81}{#1}}
\newcommand{\FloatTok}[1]{\textcolor[rgb]{0.00,0.00,0.81}{#1}}
\newcommand{\ConstantTok}[1]{\textcolor[rgb]{0.00,0.00,0.00}{#1}}
\newcommand{\CharTok}[1]{\textcolor[rgb]{0.31,0.60,0.02}{#1}}
\newcommand{\SpecialCharTok}[1]{\textcolor[rgb]{0.00,0.00,0.00}{#1}}
\newcommand{\StringTok}[1]{\textcolor[rgb]{0.31,0.60,0.02}{#1}}
\newcommand{\VerbatimStringTok}[1]{\textcolor[rgb]{0.31,0.60,0.02}{#1}}
\newcommand{\SpecialStringTok}[1]{\textcolor[rgb]{0.31,0.60,0.02}{#1}}
\newcommand{\ImportTok}[1]{#1}
\newcommand{\CommentTok}[1]{\textcolor[rgb]{0.56,0.35,0.01}{\textit{#1}}}
\newcommand{\DocumentationTok}[1]{\textcolor[rgb]{0.56,0.35,0.01}{\textbf{\textit{#1}}}}
\newcommand{\AnnotationTok}[1]{\textcolor[rgb]{0.56,0.35,0.01}{\textbf{\textit{#1}}}}
\newcommand{\CommentVarTok}[1]{\textcolor[rgb]{0.56,0.35,0.01}{\textbf{\textit{#1}}}}
\newcommand{\OtherTok}[1]{\textcolor[rgb]{0.56,0.35,0.01}{#1}}
\newcommand{\FunctionTok}[1]{\textcolor[rgb]{0.00,0.00,0.00}{#1}}
\newcommand{\VariableTok}[1]{\textcolor[rgb]{0.00,0.00,0.00}{#1}}
\newcommand{\ControlFlowTok}[1]{\textcolor[rgb]{0.13,0.29,0.53}{\textbf{#1}}}
\newcommand{\OperatorTok}[1]{\textcolor[rgb]{0.81,0.36,0.00}{\textbf{#1}}}
\newcommand{\BuiltInTok}[1]{#1}
\newcommand{\ExtensionTok}[1]{#1}
\newcommand{\PreprocessorTok}[1]{\textcolor[rgb]{0.56,0.35,0.01}{\textit{#1}}}
\newcommand{\AttributeTok}[1]{\textcolor[rgb]{0.77,0.63,0.00}{#1}}
\newcommand{\RegionMarkerTok}[1]{#1}
\newcommand{\InformationTok}[1]{\textcolor[rgb]{0.56,0.35,0.01}{\textbf{\textit{#1}}}}
\newcommand{\WarningTok}[1]{\textcolor[rgb]{0.56,0.35,0.01}{\textbf{\textit{#1}}}}
\newcommand{\AlertTok}[1]{\textcolor[rgb]{0.94,0.16,0.16}{#1}}
\newcommand{\ErrorTok}[1]{\textcolor[rgb]{0.64,0.00,0.00}{\textbf{#1}}}
\newcommand{\NormalTok}[1]{#1}
\usepackage{graphicx,grffile}
\makeatletter
\def\maxwidth{\ifdim\Gin@nat@width>\linewidth\linewidth\else\Gin@nat@width\fi}
\def\maxheight{\ifdim\Gin@nat@height>\textheight\textheight\else\Gin@nat@height\fi}
\makeatother
% Scale images if necessary, so that they will not overflow the page
% margins by default, and it is still possible to overwrite the defaults
% using explicit options in \includegraphics[width, height, ...]{}
\setkeys{Gin}{width=\maxwidth,height=\maxheight,keepaspectratio}
\IfFileExists{parskip.sty}{%
\usepackage{parskip}
}{% else
\setlength{\parindent}{0pt}
\setlength{\parskip}{6pt plus 2pt minus 1pt}
}
\setlength{\emergencystretch}{3em}  % prevent overfull lines
\providecommand{\tightlist}{%
  \setlength{\itemsep}{0pt}\setlength{\parskip}{0pt}}
\setcounter{secnumdepth}{0}
% Redefines (sub)paragraphs to behave more like sections
\ifx\paragraph\undefined\else
\let\oldparagraph\paragraph
\renewcommand{\paragraph}[1]{\oldparagraph{#1}\mbox{}}
\fi
\ifx\subparagraph\undefined\else
\let\oldsubparagraph\subparagraph
\renewcommand{\subparagraph}[1]{\oldsubparagraph{#1}\mbox{}}
\fi

%%% Use protect on footnotes to avoid problems with footnotes in titles
\let\rmarkdownfootnote\footnote%
\def\footnote{\protect\rmarkdownfootnote}

%%% Change title format to be more compact
\usepackage{titling}

% Create subtitle command for use in maketitle
\newcommand{\subtitle}[1]{
  \posttitle{
    \begin{center}\large#1\end{center}
    }
}

\setlength{\droptitle}{-2em}

  \title{Lab 7: Clustering in R}
    \pretitle{\vspace{\droptitle}\centering\huge}
  \posttitle{\par}
    \author{}
    \preauthor{}\postauthor{}
    \date{}
    \predate{}\postdate{}
  

\begin{document}
\maketitle

\begin{Shaded}
\begin{Highlighting}[]
\NormalTok{iris_data <-}\StringTok{ }\KeywordTok{subset}\NormalTok{(datasets}\OperatorTok{::}\NormalTok{iris, }\DataTypeTok{select=}\KeywordTok{c}\NormalTok{(}\DecValTok{1}\OperatorTok{:}\DecValTok{4}\NormalTok{))}
\NormalTok{target <-}\StringTok{ }\NormalTok{datasets}\OperatorTok{::}\NormalTok{iris}\OperatorTok{$}\NormalTok{Species}

\NormalTok{get_external_metrics <-}\StringTok{ }\ControlFlowTok{function}\NormalTok{(computed, target) \{}
\NormalTok{  true_positives <-}\StringTok{ }\DecValTok{0}
\NormalTok{  true_negatives <-}\StringTok{ }\DecValTok{0}
\NormalTok{  false_positives <-}\StringTok{ }\DecValTok{0}
\NormalTok{  false_negatives <-}\StringTok{ }\DecValTok{0} 
  
  \CommentTok{# for every pair of every item}
  \ControlFlowTok{for}\NormalTok{ (i }\ControlFlowTok{in} \DecValTok{1}\OperatorTok{:}\KeywordTok{length}\NormalTok{(target)) \{}
    \ControlFlowTok{for}\NormalTok{ (j }\ControlFlowTok{in} \DecValTok{1}\OperatorTok{:}\KeywordTok{length}\NormalTok{(target)) \{}
      \ControlFlowTok{if}\NormalTok{ (computed[i] }\OperatorTok{==}\StringTok{ }\NormalTok{computed[j] }\OperatorTok{&&}\StringTok{ }\NormalTok{target[i] }\OperatorTok{==}\StringTok{ }\NormalTok{target[j]) \{}
\NormalTok{        true_positives <-}\StringTok{ }\NormalTok{true_positives }\OperatorTok{+}\StringTok{ }\DecValTok{1}
\NormalTok{      \}}
      \ControlFlowTok{else} \ControlFlowTok{if}\NormalTok{ (computed[i] }\OperatorTok{!=}\StringTok{ }\NormalTok{computed[j] }\OperatorTok{&&}\StringTok{ }\NormalTok{target[i] }\OperatorTok{!=}\StringTok{ }\NormalTok{target[j]) \{}
\NormalTok{        true_negatives <-}\StringTok{ }\NormalTok{true_negatives }\OperatorTok{+}\StringTok{ }\DecValTok{1}
\NormalTok{      \}}
      \ControlFlowTok{else} \ControlFlowTok{if}\NormalTok{ (computed[i] }\OperatorTok{==}\StringTok{ }\NormalTok{computed[j] }\OperatorTok{&&}\StringTok{ }\NormalTok{target[i] }\OperatorTok{!=}\StringTok{ }\NormalTok{target[j]) \{}
\NormalTok{        false_positives <-}\StringTok{ }\NormalTok{false_positives }\OperatorTok{+}\StringTok{ }\DecValTok{1}
\NormalTok{      \}}
      \ControlFlowTok{else} \ControlFlowTok{if}\NormalTok{ (computed[i] }\OperatorTok{!=}\StringTok{ }\NormalTok{computed[j] }\OperatorTok{&&}\StringTok{ }\NormalTok{target[i] }\OperatorTok{==}\StringTok{ }\NormalTok{target[j]) \{}
\NormalTok{        false_negatives <-}\StringTok{ }\NormalTok{false_negatives }\OperatorTok{+}\StringTok{ }\DecValTok{1}
\NormalTok{      \}}
\NormalTok{    \}}
\NormalTok{  \}}
  
\NormalTok{  precision =}\StringTok{ }\NormalTok{true_positives }\OperatorTok{/}\StringTok{ }\NormalTok{(true_positives }\OperatorTok{+}\StringTok{ }\NormalTok{false_positives)}
\NormalTok{  recall =}\StringTok{ }\NormalTok{true_positives }\OperatorTok{/}\StringTok{ }\NormalTok{(true_positives }\OperatorTok{+}\StringTok{ }\NormalTok{false_negatives)}
\NormalTok{  f_score =}\StringTok{ }\NormalTok{(}\DecValTok{2} \OperatorTok{*}\StringTok{ }\NormalTok{precision }\OperatorTok{*}\StringTok{ }\NormalTok{recall) }\OperatorTok{/}\StringTok{ }\NormalTok{(precision }\OperatorTok{+}\StringTok{ }\NormalTok{recall)}
  \KeywordTok{return}\NormalTok{(}\KeywordTok{list}\NormalTok{(}\DataTypeTok{precision=}\NormalTok{precision, }\DataTypeTok{recall=}\NormalTok{recall, }\DataTypeTok{f_score=}\NormalTok{f_score))}
\NormalTok{\}}

\NormalTok{print_metrics <-}\StringTok{ }\ControlFlowTok{function}\NormalTok{(metrics) \{}
  \KeywordTok{cat}\NormalTok{(}\StringTok{"}\CharTok{\textbackslash{}n}\StringTok{Precision:"}\NormalTok{, metrics}\OperatorTok{$}\NormalTok{precision)}
  \KeywordTok{cat}\NormalTok{(}\StringTok{"}\CharTok{\textbackslash{}n}\StringTok{Recall:"}\NormalTok{, metrics}\OperatorTok{$}\NormalTok{recall)}
  \KeywordTok{cat}\NormalTok{(}\StringTok{"}\CharTok{\textbackslash{}n}\StringTok{F-score:"}\NormalTok{,metrics}\OperatorTok{$}\NormalTok{f_score)}
\NormalTok{\}}
\end{Highlighting}
\end{Shaded}

\subsubsection{Question 3: Kmeans}\label{question-3-kmeans}

3a,b. Run kmeans for selected k values and report the cluster size and
F-score.

\begin{Shaded}
\begin{Highlighting}[]
\ControlFlowTok{for}\NormalTok{ (k }\ControlFlowTok{in} \KeywordTok{c}\NormalTok{(}\DecValTok{2}\NormalTok{,}\DecValTok{3}\NormalTok{,}\DecValTok{4}\NormalTok{,}\DecValTok{5}\NormalTok{,}\DecValTok{7}\NormalTok{,}\DecValTok{9}\NormalTok{,}\DecValTok{11}\NormalTok{)) \{}
\NormalTok{  clusters <-}\StringTok{ }\KeywordTok{kmeans}\NormalTok{(iris_data, k)}
\NormalTok{  metrics <-}\StringTok{ }\KeywordTok{get_external_metrics}\NormalTok{(clusters}\OperatorTok{$}\NormalTok{cluster, target)}
  
  \KeywordTok{cat}\NormalTok{(}\StringTok{"K value:"}\NormalTok{, k)}
  \KeywordTok{cat}\NormalTok{(}\StringTok{"}\CharTok{\textbackslash{}n}\StringTok{Cluster sizes:"}\NormalTok{, clusters}\OperatorTok{$}\NormalTok{size)}
  \KeywordTok{print_metrics}\NormalTok{(metrics)}
  \KeywordTok{cat}\NormalTok{(}\StringTok{"}\CharTok{\textbackslash{}n\textbackslash{}n}\StringTok{"}\NormalTok{)}
\NormalTok{\}}
\end{Highlighting}
\end{Shaded}

\begin{verbatim}
## K value: 2
## Cluster sizes: 97 53
## Precision: 0.5907677
## Recall: 0.9624
## F-score: 0.7321229
## 
## K value: 3
## Cluster sizes: 62 38 50
## Precision: 0.8089368
## Recall: 0.84
## F-score: 0.8241758
## 
## K value: 4
## Cluster sizes: 28 45 50 27
## Precision: 0.8234515
## Recall: 0.6629333
## F-score: 0.734525
## 
## K value: 5
## Cluster sizes: 37 12 50 24 27
## Precision: 0.8728845
## Recall: 0.6189333
## F-score: 0.724294
## 
## K value: 7
## Cluster sizes: 37 27 33 23 7 6 17
## Precision: 0.8347188
## Recall: 0.4552
## F-score: 0.5891286
## 
## K value: 9
## Cluster sizes: 9 12 50 9 12 16 12 8 22
## Precision: 0.9830682
## Recall: 0.5109333
## F-score: 0.6723987
## 
## K value: 11
## Cluster sizes: 17 4 10 33 9 12 16 18 22 5 4
## Precision: 0.9752125
## Recall: 0.3672
## F-score: 0.5335141
\end{verbatim}

3c. The value of \texttt{k} that produces the highest F-score is 3, with
an F-score of \texttt{0.824}.

3d. This is to be expected, since there are 3 original species in the
dataset.

\subsubsection{Question 4: hclust}\label{question-4-hclust}

4a. Display the result of the \texttt{hclust} algorithm as a dendrogram.

\begin{Shaded}
\begin{Highlighting}[]
\NormalTok{iris_dist <-}\StringTok{ }\KeywordTok{dist}\NormalTok{(iris_data)}
\NormalTok{clusters <-}\StringTok{ }\KeywordTok{hclust}\NormalTok{(iris_dist)}
\KeywordTok{plot}\NormalTok{(clusters)}
\end{Highlighting}
\end{Shaded}

\includegraphics{lab7_files/figure-latex/unnamed-chunk-3-1.pdf}

4b. Looking at the display, the optimal clustering threshold I see is 2
clusters, since the height of the 2 clusters is greatest (between 7 and
4). This is interesting because the actual number of clusters is 3, but
2 of those 3 are super close together and easily mistaken for each
other. The height for 3 clusters is a super small gap between 3.5 and 4,
which doesn't seem very optimal at all.

4c. The \texttt{kmeans} algorithm picked up on the fact that there are 3
clusters, but the \texttt{hclust} algorithm shows that there are only 2.
The composition of the 3 clusters shows that 1 is separate, and the
other 2 are very close together. \texttt{hclust} doesn't distinguish
between those 2 close clusters.

\subsubsection{Question 5: dbscan}\label{question-5-dbscan}

5a,b. Run dbscan for selected eps and report the cluster size and
F-score.

\begin{Shaded}
\begin{Highlighting}[]
\NormalTok{iris_matrix <-}\StringTok{ }\KeywordTok{as.matrix}\NormalTok{(iris_data)}
\ControlFlowTok{for}\NormalTok{ (eps }\ControlFlowTok{in} \KeywordTok{c}\NormalTok{(}\FloatTok{0.2}\NormalTok{, }\FloatTok{0.3}\NormalTok{, }\FloatTok{0.4}\NormalTok{, }\FloatTok{0.5}\NormalTok{, }\FloatTok{0.6}\NormalTok{, }\FloatTok{0.8}\NormalTok{, }\FloatTok{1.0}\NormalTok{)) \{}
\NormalTok{  clusters <-}\StringTok{ }\NormalTok{dbscan}\OperatorTok{::}\KeywordTok{dbscan}\NormalTok{(iris_matrix, eps)}
\NormalTok{  metrics <-}\StringTok{ }\KeywordTok{get_external_metrics}\NormalTok{(clusters}\OperatorTok{$}\NormalTok{cluster, target)}
  
  \KeywordTok{cat}\NormalTok{(}\StringTok{"eps value:"}\NormalTok{, eps)}
  \KeywordTok{cat}\NormalTok{(}\StringTok{"}\CharTok{\textbackslash{}n}\StringTok{Cluster sizes:"}\NormalTok{, }\KeywordTok{as.data.frame}\NormalTok{(}\KeywordTok{table}\NormalTok{(clusters}\OperatorTok{$}\NormalTok{cluster))[,}\DecValTok{2}\NormalTok{])}
  \KeywordTok{print_metrics}\NormalTok{(metrics)}
  \KeywordTok{cat}\NormalTok{(}\StringTok{"}\CharTok{\textbackslash{}n\textbackslash{}n}\StringTok{"}\NormalTok{)}
\NormalTok{\}}
\end{Highlighting}
\end{Shaded}

\begin{verbatim}
## eps value: 0.2
## Cluster sizes: 133 10 7
## Precision: 0.3497029
## Recall: 0.8317333
## F-score: 0.492383
## 
## eps value: 0.3
## Cluster sizes: 96 37 12 5
## Precision: 0.4820532
## Recall: 0.6912
## F-score: 0.5679851
## 
## eps value: 0.4
## Cluster sizes: 32 46 36 14 22
## Precision: 0.8702111
## Recall: 0.5936
## F-score: 0.7057705
## 
## eps value: 0.5
## Cluster sizes: 17 49 84
## Precision: 0.62323
## Recall: 0.8098667
## F-score: 0.7043952
## 
## eps value: 0.6
## Cluster sizes: 9 49 92
## Precision: 0.6089896
## Recall: 0.8888
## F-score: 0.7227583
## 
## eps value: 0.8
## Cluster sizes: 2 50 98
## Precision: 0.6035679
## Recall: 0.9744
## F-score: 0.74541
## 
## eps value: 1
## Cluster sizes: 50 100
## Precision: 0.6
## Recall: 1
## F-score: 0.75
\end{verbatim}

5c. The value of \texttt{eps} that produces the highest F-score is
\texttt{1.0}, with an F-score of \texttt{0.75}.

5d. This is an interesting result, since this value of \texttt{eps}
actually only recognizes 2 clusters instead of the 3 that are actually
present. Other \texttt{eps} values experimented with different numbers
of clusters, and a few had 3 clusters but not a high enough F-score. A
recall value of \texttt{1} means that there were no false negatives
identified with \texttt{eps=1}, which is interesting and a bit worrying
to me.

\subsubsection{Question 6: swiss}\label{question-6-swiss}

\begin{Shaded}
\begin{Highlighting}[]
\NormalTok{swiss_data <-}\StringTok{ }\KeywordTok{subset}\NormalTok{(datasets}\OperatorTok{::}\NormalTok{swiss, }\DataTypeTok{select=}\KeywordTok{c}\NormalTok{(}\DecValTok{1}\NormalTok{,}\DecValTok{2}\NormalTok{,}\DecValTok{3}\NormalTok{,}\DecValTok{4}\NormalTok{))}
\NormalTok{clusters <-}\StringTok{ }\KeywordTok{kmeans}\NormalTok{(swiss_data, }\DecValTok{2}\NormalTok{, }\DataTypeTok{nstart =} \DecValTok{5}\NormalTok{)}
\NormalTok{protestant =}\StringTok{ }\KeywordTok{names}\NormalTok{(}\KeywordTok{which}\NormalTok{(clusters}\OperatorTok{$}\NormalTok{cluster }\OperatorTok{==}\StringTok{ }\DecValTok{2}\NormalTok{))}
\NormalTok{catholic =}\StringTok{ }\KeywordTok{names}\NormalTok{(}\KeywordTok{which}\NormalTok{(clusters}\OperatorTok{$}\NormalTok{cluster }\OperatorTok{==}\StringTok{ }\DecValTok{1}\NormalTok{))}
\end{Highlighting}
\end{Shaded}

A list of the Swiss cities that are predominantly Protestant:

\begin{Shaded}
\begin{Highlighting}[]
\NormalTok{protestant}
\end{Highlighting}
\end{Shaded}

\begin{verbatim}
##  [1] "Delemont"     "Franches-Mnt" "Moutier"      "Neuveville"  
##  [5] "Porrentruy"   "Broye"        "Glane"        "Gruyere"     
##  [9] "Sarine"       "Veveyse"      "Aigle"        "Aubonne"     
## [13] "Avenches"     "Cossonay"     "Echallens"    "Lavaux"      
## [17] "Morges"       "Moudon"       "Nyone"        "Orbe"        
## [21] "Oron"         "Payerne"      "Paysd'enhaut" "Rolle"       
## [25] "Yverdon"      "Conthey"      "Entremont"    "Herens"      
## [29] "Martigwy"     "Monthey"      "St Maurice"   "Sierre"      
## [33] "Sion"         "Val de Ruz"
\end{verbatim}

A list of the Swiss cities that are predominantly Catholic:

\begin{Shaded}
\begin{Highlighting}[]
\NormalTok{catholic}
\end{Highlighting}
\end{Shaded}

\begin{verbatim}
##  [1] "Courtelary"   "Grandson"     "Lausanne"     "La Vallee"   
##  [5] "Vevey"        "Boudry"       "La Chauxdfnd" "Le Locle"    
##  [9] "Neuchatel"    "ValdeTravers" "V. De Geneve" "Rive Droite" 
## [13] "Rive Gauche"
\end{verbatim}


\end{document}
